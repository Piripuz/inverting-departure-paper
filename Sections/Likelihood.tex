\section{Likelihood}

In the current section,
the parameters \(\beta, \gamma, t^*\) will be considered to be distributed, according to some known distribution.
This makes the outcomes of the function \(t^{opt}(\beta, \gamma, t^*)\) defined above to be distributed themselves.

\subsection{Theoretical Prerequisites}


The goal of the section is thus to find an analytical expression for the Probability Density Function (pdf) that characterizes the distribution of the outcomes of the function \(t^{opt}\).
To this end, the characterization developed in the preceding section will be employed,
in a slightly simpler setting, in which some additional hypotheses on the travel time function are made.

The assumptions that simplify the estimation are shown in the following definition:
\begin{definition}
  \label{def:proper_tt}
  Let \(tt_a:\R\rightarrow\R\).

  \(tt_a\) is a \textbf{Proper Travel Time Function} if \(tt_a\) is a General Travel Time Function and,
  on top of this, 
  two points \(k_1\neq k_2 \in \R\) can be found such that
  \begin{itemize}
  \item \(tt_a|_{(k_1, k_2)}\) is concave
  \item \(tt_a|_{\R\setminus(k_1, k_2)}\) is convex
  \end{itemize}
\end{definition}
\todo{Why is proper tt even realistic?}

For a proper travel time function,
the estimation of the optimal arrival time \(t^{opt}\) considerably simplifies:
the following observation shows how.
\begin{obs}
  Consider the problem of choosing an optimal departure time
  with a proper travel time function, and let \(\beta, \gamma\) be fixed.

  There exist two intervals \((t_e, \bar{t_e}), (\bar{t_l}, t_l)\) such that an on-time arrival \(t^{opt} = t^*\) is realized if and only if
  \begin{equation*}
    t^* \notin (t_e, \bar{t_e}) \cup (\bar{t_l}, t_l)
  \end{equation*}
\end{obs}

Note that this is exactly what said by Proposition~\ref{prop:into-early-late},
with a single difference:
the intervals are indeed here at most one (for each type of arrival), rather than an arbitrary number.

This is a consequence of the hypothesis on the travel time function:
the point \(t_e\) must indeed be an eligible early arrival and,
as shown in Lemma~\ref{lemma:bounded-der-tt} and~\ref{lemma:cost_decoupled},
any possible early arrival must satisfy two simple conditions:
\begin{enumerate}
\item \(tt_a'(t_e) = \beta\)
\item \(tt_a''(t_e) \geq 0\)
\end{enumerate}
where the second one derives from being a minimum:
if the second derivative was negative, the point was indeed a maximum.

But for a proper travel time function,
a point satisfying the conditions above can occur only once:
given the unimodality,
the function will be as shown in Figure\todo{ref}: increasing, with an increasing derivative,
for \(t < k_1\), and decreasing with an increasing derivative for \(t > k_2\).
Its derivative will thus be injective on the convex part \(\R \setminus (k_1, k_2)\),
that is the part we are interested in.
This shows that the set of possible \(t^*\) for which early arrivals occur is a single interval.
A similar reasoning holds for late arrivals.

Consider thus now the problem of finding the intervals \((t_e, \bar{t_e})\), \((\bar{t_l}, t_l)\).
The points in which the arrivals occur are now easy to estimate. Let
\begin{equation*}
  \beta_{max} = \max_t tt_a'(t)\qquad \gamma_{max} = -\min_t tt_a'(t)
\end{equation*}

The following functions are well defined:
\begin{align*}
  b_i: (0, \beta_{max}) & \rightarrow \R  & g_e: (0, \gamma_{max}) & \rightarrow \R \\
       \beta & \mapsto (tt_a' |_{(-\infty, k_1)})^{-1}(\beta) & \gamma & \mapsto(tt_a' |_{(k_2, \infty)})^{-1}(\gamma)
\end{align*}

Given \(\beta \in (0, \beta_{max})\),
the time \(b_i(\beta)\) will be the starting point of the interval in which early arrivals occur.

Similarly, given \(\gamma\) the time \(g_e(\gamma)\) will be the ending point of the interval in which the late arrivals occur.

Estimating the remaining point is slightly more difficult.
The expression in the proof of Proposition~\ref{prop:into-early-late} can anyway be used:
the function \(b_e(\beta)\) (and \(g_i(\gamma)\))
are thus defined as the points in which the line tangent to the travel time function in \(b_i(\beta)\) (or, respectively, in \(g_e(\gamma)\)) intersects again with the travel time function.

A graphical representation of the functions \(b_i, b_e, g_i, g_e\) is shown in figure\todo{ref}
\todo{Here, how \(t_s\) is computed has to be explained. These things are to be put in this or in the theory section?? Maybe theory}
Once these functions have been defined,
the pdf that characterizes the distribution of the optimal departure time \(t^{opt}\) can be computed.

\subsection{The Probability Density Function}

Consider now the same problem in which,
instead of the parameters \(\beta, \gamma, t^*\),
we consider three random variables \(B, \Gamma, T^*\).

A random variable for the optimal arrival will be defined as follows:
\begin{equation}
  \label{eq:rv-opt-arr}
  T^{opt} = t^{opt}(B, \Gamma, T^*)
\end{equation}

The goal of this section will be finding the pdf \(f_{t^{opt}}\) of the variable \(T^{opt}\).

It is, in the first place,
convenient to distinguish between early, late and on-time arrivals.
In order to distinguish them, we define another random variable:
\begin{definition}
  Consider the function
  \begin{equation*}
    q(\beta, \gamma, t^*) =
    \begin{cases}
      -1 & \text{if } t^{opt}(\beta, \gamma, t^*) = t_e^{opt}(\beta, \gamma, t^*) \\
      0 & \text{if } t^{opt}(\beta, \gamma, t^*) = t^* \\
      1 & \text{if } t^{opt}(\beta, \gamma, t^*) = t_l^{opt}(\beta, \gamma, t^*)
    \end{cases}
  \end{equation*}

  Given the random variables \(B, \Gamma, T^*\), another random variable \(Q\) is defined as follows:
  \begin{equation*}
    Q  = q(B, \Gamma, T^*)
  \end{equation*}
\end{definition}

In practical terms, the random variable \(Q\) shows whether an early, on-time or late arrival is realized,
and takes the value \(-1\) if the optimal arrival is an early arrival,
\(1\) if it is a late arrival and \(0\) if it is an on-time one.

%%% Local Variables:
%%% mode: LaTeX
%%% TeX-master: "../main"
%%% End:
