\section{Likelihood}

In the current section,
the parameters \(\beta, \gamma, t^*\) will be considered to be distributed, according to some known distribution.
This makes the outcomes of the function \(t^{opt}(\beta, \gamma, t^*)\) defined above to be distributed themselves.

\subsection{Theoretical Prerequisites}


The goal of the section is thus to find an analytical expression for the Probability Density Function (pdf)\todo{Is it ok to write pdf or is it informal?} that characterizes the distribution of the outcomes of the function \(t^{opt}\).
To this end, the characterization developed in the preceding section will be employed,
in a slightly simpler setting, in which some additional hypotheses on the travel time function are made.

The assumptions that simplify the estimation are shown in the following definition:
\begin{definition}
  \label{def:proper_tt}
  Let \(tt_a:\R\rightarrow\R\).

  \(tt_a\) is a \textbf{Proper Travel Time Function} if \(tt_a\) is a General Travel Time Function and,
  on top of this, 
  two points \(k_1\neq k_2 \in \R\) can be found such that
  \begin{itemize}
  \item \(tt_a|_{(k_1, k_2)}\) is concave
  \item \(tt_a|_{\R\setminus(k_1, k_2)}\) is convex
  \end{itemize}
\end{definition}
\todo{Why is proper tt even realistic?}

For a proper travel time function,
the estimation of the optimal arrival time \(t^{opt}\) considerably simplifies:
the following observation shows how.
\begin{obs}
  \label{obs:simplified-char}
  Consider the problem of choosing an optimal departure time
  with a proper travel time function, and let \(\beta, \gamma\) be fixed.

  There exist two intervals \((t_e, \bar{t_e}), (\bar{t_l}, t_l)\) such that an on-time arrival \(t^{opt} = t^*\) is realized if and only if
  \begin{equation*}
    t^* \notin (t_e, \bar{t_e}) \cup (\bar{t_l}, t_l)
  \end{equation*}
\end{obs}

Note that this is implied from what said by Proposition~\ref{prop:into-early-late},
except for a single detail:
the intervals are indeed here at most one (for each type of arrival), rather than an arbitrary number.

This is a consequence of the hypothesis on the travel time function:
the point \(t_e\) must indeed be an eligible early arrival and,
as shown in Lemma~\ref{lemma:bounded-der-tt} and~\ref{lemma:cost_decoupled},
any possible early arrival must satisfy two simple conditions:
\begin{enumerate}
\item \(tt_a'(t_e) = \beta\)
\item \(tt_a''(t_e) \geq 0\)
\end{enumerate}
where the second one derives from being a minimum:
if the second derivative was negative, the point was indeed a maximum.

But for a proper travel time function,
a point satisfying the conditions above can occur only once:
given the unimodality,
the function will be as shown in Figure\todo{ref}: increasing, with an increasing derivative,
for \(t < k_1\), and decreasing with an increasing derivative for \(t > k_2\).
Its derivative will thus be injective on the convex part \(\R \setminus (k_1, k_2)\),
that is the part we are interested in.
This shows that the set of possible \(t^*\) for which early arrivals occur is a single interval.
A similar reasoning holds for late arrivals.

Consider thus now the problem of finding the intervals \((t_e, \bar{t_e})\), \((\bar{t_l}, t_l)\).
The points in which the arrivals occur are now easy to estimate. Let
\begin{equation*}
  \beta_{max} = \max_t tt_a'(t)\qquad \gamma_{max} = -\min_t tt_a'(t)
\end{equation*}

The following functions are well defined:
\begin{align*}
  b_i: (0, \beta_{max}) & \rightarrow (-\infty, k_1)  & g_e: (0, \gamma_{max}) & \rightarrow (k_2, \infty) \\
       \beta & \mapsto (tt_a' |_{(-\infty, k_1)})^{-1}(\beta) & \gamma & \mapsto(tt_a' |_{(k_2, \infty)})^{-1}(\gamma)
\end{align*}

Given \(\beta \in (0, \beta_{max})\),
the time \(b_i(\beta)\) will be the starting point of the interval in which early arrivals occur.

Similarly, given \(\gamma\) the time \(g_e(\gamma)\) will be the ending point of the interval in which the late arrivals occur.

Estimating the remaining point is slightly more difficult.
The expression in the proof of Proposition~\ref{prop:into-early-late} can anyway be used:
the function \(b_e(\beta)\) (and \(g_i(\gamma)\))
are thus defined as the points in which the line tangent to the travel time function in \(b_i(\beta)\) (or, respectively, in \(g_e(\gamma)\)) intersects again with the travel time function.

A graphical representation of the functions \(b_i, b_e, g_i, g_e\) is shown in figure\todo{ref}
\todo{Here, how \(t_s\) is computed has to be explained. These things are to be put in this or in the theory section?? Maybe theory}
Once these functions have been defined,
the pdf that characterizes the distribution of the optimal departure time \(t^{opt}\) can be computed.

\subsection{The Probability Density Function}

Consider now the same problem in which,
instead of the parameters \(\beta, \gamma, t^*\),
we consider three random variables \(B, \Gamma, T^*\).

In order to simplify the estimation process, an assumption (which is, indeed, quite restrictive)
is made on these random variable:

\begin{assumption}
  The random variables \(B, \Gamma, T^*\) are pairwise independent
\end{assumption}

Note that this assumption is rather significant,
but not as significant as it may sound:
remind indeed that, being the parameter \(\alpha\) in equation~\eqref{eq:cost_ta} normalized to 1,
the parameters \(\beta, \gamma\) represent the normalized values of the early and late arrival penalties instead of the actual ones.

A random variable for the optimal arrival will be defined as follows:
\begin{equation}
  \label{eq:rv-opt-arr}
  T^{opt} = t^{opt}(B, \Gamma, T^*)
\end{equation}

The goal of this section will be finding the pdf \(f_{t^{opt}}\) of the variable \(T^{opt}\).

It is, in the first place,
convenient to distinguish between early, late and on-time arrivals.
In order to distinguish them, we define another random variable
\begin{definition}
  Consider the function
  \begin{equation*}
    q(\beta, \gamma, t^*) =
    \begin{cases}
      -1 & \text{if } t^{opt}(\beta, \gamma, t^*) = t_e^{opt}(\beta, \gamma, t^*) \\
      0 & \text{if } t^{opt}(\beta, \gamma, t^*) = t^* \\
      1 & \text{if } t^{opt}(\beta, \gamma, t^*) = t_l^{opt}(\beta, \gamma, t^*)
    \end{cases}
  \end{equation*}

  Given the random variables \(B, \Gamma, T^*\), the random variable \(Q\) is defined as follows:
  \begin{equation*}
    Q  = q(B, \Gamma, T^*)
  \end{equation*}
\end{definition}

In practical terms, the random variable \(Q\) shows whether an early, on-time or late arrival is realized,
and takes the value \(-1\) if the optimal arrival is an early arrival,
\(1\) if it is a late arrival and \(0\) if it is an on-time one.

Since an optimal arrival can only be early, on-time or late,
by using the sum rule, the pdf for the random variable \(T^{opt}\) can be decomposed into three different parts:
\begin{equation}
  \label{eq:pdf-decomposed-q}
  f_{t^{opt}}(t) = f_{T^{opt}, Q}(t, -1) + f_{T^{opt}, Q}(t, 0) + f_{T^{opt}, Q}(t, 1)
\end{equation}

The single addends are easier to describe,
and will be individually studied in the following.

\subsubsection{On-time Arrivals}

How likely a point is to be an on-time arrival will be here computed.
This is, the value of the function
\begin{equation*}
  f_{T^{opt}, Q}(t, 0)
\end{equation*}

Note that Observation~\ref{obs:simplified-char} fully characterizes the values of the parameters for which the on-time arrivals are realized:
we are able, by using this characterization, to better describe the event at issue.

In particular, we note that the arrival time \(t\) is an optimal on-time arrival if and only if two conditions are satisfied:
the time \(t\) has to be a realization of the desired arrival time \(T^*\),
and it has to  be out from the intervals \((b_i(B), b_e(B)), (g_i(\Gamma), g_e(\Gamma))\).

Being the random variable \(T^*\) independent from \(B, \Gamma\),
the two events above are independent and their likelihood can be multiplied:
\begin{equation}
  \label{eq:on-time-intro}
  f_{T^{opt}, Q}(t, 0) = f_{T^*}(t) \cdot \prob(t \notin (b_i(B), b_e(B)) \cup (g_i(\Gamma), g_e(\Gamma)))
\end{equation}
where \(f_{T^*}\) is the pdf of the random variable \(T^*\), and is known.

Expression~\eqref{eq:on-time-intro} leads to a straightforward way of computing the desired value:
the probability of being out from the intervals \((b_i(B), b_e(B)), (g_i(\Gamma), g_e(\Gamma))\) can indeed be simply evaluated by integrating it

\begin{equation}
  \label{eq:prob-not-intervals}
  \prob( t \not\in (b_i(B), b_e(B)) \cup (g_i(\Gamma), g_e(\Gamma))) = \int_{b\in \R \vert t \not\in (b_i(b), b_e(b))}\int_{g \in \R \vert t \not\in (g_i(g), g_e(g))}f_B(b)f_\Gamma(g)\, dg\, db
\end{equation}
where \(f_B, f_\Gamma\) are the pdf of, respectively, the random variables \(B, \Gamma\).


It is now possible to take advantage of Proposition\todo{Need to add this proposition somewhere...}:
the monotonicity of the intervals imply that there exist two functions representing the maximum values \(\beta\) and \(\gamma\) can assume while keeping a given point into the defined intervals:
\begin{equation}
  \label{eq:def_b_0_g_0}
  \begin{split}
    \beta_0:\R&\rightarrow(0, \beta_{max}) \\
    t&\mapsto \sup\{b | t \in (b_i(b), b_e(b))\} \\[1em]
    \gamma_0:\R&\rightarrow(0, \gamma_{max}) \\
    t&\mapsto \sup\{g | t \in (g_i(g), g_e(g))\}
  \end{split}
\end{equation}

The definition of these functions simplifies the integrals in~\eqref{eq:prob-not-intervals}, that become

\begin{equation}
  \label{eq:on-time-monot}
  \prob( t \not\in (b_i(B), b_e(B)) \cup (g_i(\Gamma), g_e(\Gamma))) = \int_{\beta_0(t)}^{\beta_{\text{max}}}\int_{\gamma_0(t)}^{\gamma_{\text{max}}}f_B(b)f_\Gamma(g)\, dg\, db
\end{equation}

By exploiting the expression obtained in equation~\eqref{eq:on-time-monot},
the joint pdf in equation~\eqref{eq:on-time-intro} can be analytically evaluated:
\begin{equation}
  \label{eq:on-time-final}
  \begin{split}
    f_{T^{opt}, Q}(t, 0) & = f_{T^*}(t) \cdot \prob(t \notin (b_i(B), b_e(B)) \cup (g_i(\Gamma), g_e(\Gamma))) \\
    & = f_{T^*}(t)\int_{\beta_0(t)}^{\beta_{\text{max}}}\int_{\gamma_0(t)}^{\gamma_{\text{max}}}f_B(b)f_\Gamma(g)\, dg\, db
  \end{split}
\end{equation}

Note that all the functions used in the right hand side can be evaluated.
This concludes thus the evaluation of the likelihood of being an on-time arrival.
In the following part, the likelihood of being an early or late arrival will be computed.

\subsubsection{Early and Late Arrivals}

Consider now the value of the function
\begin{equation*}
  f_{T^{opt}, Q}(t, -1)
\end{equation*}

Observation\todo{an observation on the definition of \(t_s\)} shows that the arrival time \(t\) is an early arrival if and only if two conditions,
similar to the ones in the preceding case,
are realized:
the time \(t\) has to be a realization of the optimal early arrival time \(b_i(B)\),
and it has to be into the interval \((b_i(B), \min\{b_e(B), t_s(B, \Gamma)\})\).

Differently from the last case, the random variable \(B\) appears in both conditions:
the events are thus not independent anymore,
and their joint pdf will be computed via its decomposition in conditional densities:
\begin{equation}
  \label{eq:late-joint-decomp}
  f_{T^{opt}, Q}(t, -1) = \prob\left(t^* \in (b_i(B), \min\{b_e(B), t_s(B, \Gamma)\})|\, t = b_i(B)\right) f_{b_i(B)}(t)
\end{equation}

The probability density function \(f_{b_i(B)}(t)\) for the transformed random variable \(b_i(B)\) can be simply evaluated by changing variable from the known density \(f_B\):
\begin{equation}
  \label{eq:pdf-b-change-var}
  f_{b_i(B)}(t) = f_B(b_i^{-1}(t)) \left| \diff{}{t}(b_i^{-1}(t)) \right|
\end{equation}

From the definition of \(b_i\) follows that the function acts,
on its domain, exactly as the inverse of the derivative of the travel time function \((tt_a')^{-1}\).

From this observation follows that, as long as we don't consider times that are out of the image of the function \(b_i\),
we can set
\begin{equation*}
  b_i^{-1}(t) = tt_a'(t)
\end{equation*}

The arrival times out of the image of \(b_i\) are exactly the ones for which the function is concave
\footnote{
  Depending on whether the parameter \(\beta\) is allowed to be negative,
  the interval in which the derivative of the travel time \(tt_a'(t)\) is negative can be considered out of the image as well.
  This is not considered relevant herein,
  since the case in which the parameter \(\beta\) is negative is not considered:
  the probability that a realization of the random variable \(B\) is negative will thus be neglectable.
},
that is, \(tt_a''(t) < 0\).

The pdf for the transformed random variable will thus be
\begin{equation}
  \label{eq:pdf-b-changed-var}
  f_{b_i(B)}(t) = f_B(tt_a'(t)) [tt_a''(t)]^+
\end{equation}
where the employment of the function \([\bullet]^+\) ensures the pdf to be equal to zero where the travel time function is concave.



%%% Local Variables:
%%% mode: LaTeX
%%% TeX-master: "../main"
%%% End:
