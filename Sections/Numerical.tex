\section{Numerical}

In this section, numerical experiments will be conducted to test the developed method,
and to evaluate wether it can be deployed in a real scenario.

The experiment will proceed in the following way:
in the first place, a profile for a theoretical travel time function will be defined.
This profile will need to satisfy assumptions NUMBERS\todo{Cite all the assumptions},
in order to be similar to a profile that could be found in reality.

After the function has been defined,
shapes for the distributions of the parameters \(B, \Gamma, T^*\) will be fixed,
dependant on a parameter \(\theta \in \R^d\).

Given all of this and a value \(\theta_0\) for the shape parameter \(\theta\),
a dataset of synthetic observations can be built accordingly to the model.
The dataset is constructed by sampling the parameters \(\beta, \gamma, t^*\) and explicitly minimizing the resulting cost function.
This yields a dataset for which the real value of the parameter \(\theta\),
which affects the distributions of \(B, \Gamma, T^*\), is known,
allowing the actual convergence of the estimation to be evaluated in a simple scenario,
and the behaviour of the likelihood function to be shown when the dataset is fixed.

An actual distribution for the random variables \(B, \Gamma, T^*\) has now to be assumed.

We thus consider
\begin{align*}
  B \sim \mathcal{N}(\mu_\beta, \sigma) && \Gamma \sim \mathcal{N}(\mu_\gamma, \sigma) && T^* \sim \mathcal{N}(\mu_t, \sigma_t)
\end{align*}

The parameter \(\theta\) will thus be
\begin{equation*}
  \theta =
  \begin{pmatrix}
    \mu_\beta \\
    \mu_\gamma \\
    \mu_t \\
    \sigma \\
    \sigma_t
  \end{pmatrix}
  \in \R^5
\end{equation*}

%%% Local Variables:
%%% mode: LaTeX
%%% TeX-master: "../main"
%%% End:
